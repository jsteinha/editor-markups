\usepackage[normalem]{ulem}

\newcommand{\add}[1]{{\color{blue!70!black} #1}}
\newcommand{\del}[1]{{\color{blue!50} \sout{#1}}}
\newcommand{\rep}[2]{{\color{blue!50} \sout{#1}}{\color{blue!70!black} #2}}
\newcommand{\cmt}[2]{{\color{blue!60} #1}\footnote{{\color{red!60!black} $\leftarrow$ #2}}}

\newcommand{\myst}{{\bf \color{purple} \hyperref[exp:myst]{mysterious}}}
\newcommand{\cit}{{\bf \color{green!50!black} \hyperref[exp:cit]{cite}}}
\newcommand{\wdy}{{\bf \color{orange} \hyperref[exp:wdy]{wordy}}}
\newcommand{\unimp}{{\bf \color{blue!70} \hyperref[exp:unimp]{unimportant}}}

\newcommand{\ExplanationPage}{
\newpage
\onecolumn

\section*{Explanation of editor markings}
\begin{itemize}
\item \label{exp:myst} \myst: This will probably come across as mysterious to readers who don't already have the context to know what you're talking about.
\item \label{exp:cit} \cit: You should try to cite something here (e.g. to give an example of, or justification for, what you just talked about).
\item \label{exp:wdy} \wdy: This is stated in more words than is necessary and can probably be shortened.
\item \label{exp:unimp} \unimp: This is not that important relative to the length/prominence it's currently given.
\end{itemize}
}
